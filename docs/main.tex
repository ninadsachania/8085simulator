\documentclass[11pt, a4paper, oneside]{book}
\usepackage{times}
\usepackage{graphicx}
\usepackage{amsmath}
\usepackage{amssymb}
\usepackage[T1]{fontenc}
\usepackage[latin9]{inputenc}
\usepackage{float}
\usepackage{subcaption}
\usepackage{booktabs}
\usepackage[top=2cm, left=2cm, right=2cm, bottom=2cm]{geometry}
\usepackage{xcolor}
\usepackage{url}
\usepackage{epstopdf}
\usepackage[
	plainpages=false,
	pdfauthor={Jubin Mitra},
	pdftitle={8085 Simulator},
	pdfsubject={8085 Simulator User Manual},
	pdfkeywords={},
	pdfproducer={Jubin Mitra},
	pdfcreator={pdfLaTeX},
	pdfdisplaydoctitle={true},
	pdfpagetransition={Blinds},
	linktocpage=true,
	colorlinks=true
]{hyperref}
\usepackage{cleveref}
\usepackage[
	natbib=true,
	style=ieee,
	backend=bibtex8,
	hyperref=auto,
	backref=true,
	backrefstyle=two,
]{biblatex}

\addbibresource{references.bib}

\newcommand{\ver}{2.0}

\captionsetup{labelfont=bf}

\graphicspath{ {./images} }
\apptocmd{\sloppy}{\hbadness 1394\relax}{}{}
\Crefformat{subfigure}{Sub-figure~(#2#1#3)}

\begin{document}
	\pagenumbering{alph}

	\thispagestyle{empty}

	\begin{center}
		\vspace{0.03\textheight}
		\textit{A User Manual\\}
		\vspace{0.01\textheight}
		\textit{on\\}
		\vspace{0.03\textheight}
		{\Huge\textbf{8085 Simulator }}\\ {\url{https://8085Simulator.github.io}}\\ {\url{{https://8085simulator.codeplex.com}}}\\

		\vspace{0.015\textheight}
		\textit{Product Version \ver}\\ \textsc{stable release}
		\vspace{0.015\textheight}
		\\By\\ \textbf{\textsc{JUBIN MITRA}}\\
		\begin{figure}[htbp]
			\centering
			\includegraphics[width=0.2\linewidth]{qr_code}
		\end{figure}
		\vspace{0.02\textheight}
		\begin{figure}[H]
			\centering
			\includegraphics[width=\linewidth]{"./top_page1"}
		\end{figure}
		\today
	\end{center}

	\chapter*{Version and Bug fixes}
	\section*{Release Date}
	\begin{table}[htbp]
		\resizebox{\linewidth}{!}{
		\begin{tabular}{lrc}
			\textbf{Version}                                                    & \textbf{Release Date} & \textbf{Download Link}                                                               \\
			\\
			Version 1                                                           & $10^{th}$ Oct, 2009   & \url{https://github.com/8085simulator/8085simulator/blob/master/8085Compiler_v0.jar} \\
			\multicolumn{3}{r}{\includegraphics[width=\linewidth]{./8085v1.png}} \\
			\\
			Version 2                                                           & $1^{st}$ Jun, 2014    & \url{https://github.com/8085simulator/8085simulator/blob/master/8085Compiler_v1.jar} \\
			\multicolumn{3}{r}{\includegraphics[width=\linewidth]{./8085v2.png}} \\
		\end{tabular}}
	\end{table}

	\section*{Bug fixes}
	From the 2nd Version of this software bug history log is maintained.

	\clearpage

	\pagenumbering{roman}

	\chapter*{Preface}

	This software was first published in October 10, 2009 and since then it has been
	in this field. It is gratifying to see such acceptance and popularity of the
	software in many institutes and universities. This tool is an integrated software
	environment for teaching microprocessor concepts. The second version of the
	software has undergone many changes and bug fixing.

	\section*{Migration Notice}

	This is to bring to the notice of the users that the previously popular site \url{http://8085simulator.codeplex.com}
	has been migrated to \url{https://8085Simulator.github.io}. Everything
	remained the same. Hope it would be of not much trouble for the users. The
	design still maintains the same simplicity and maintained under the open
	source license GPL v2.

	\section*{About the Author}

	Author has completed his B.Tech. in Electronics and Communication Engineering
	from Heritage Institute of Technology, Kolkata and M.E. from Bengal
	Engineering and Science University (BESU), Howrah, India. He is currently pursuing
	Ph.D. at Variable Energy Cyclotron Centre (VECC) at Kolkata under the aegis of
	Homi Bhabha National Institute (HBNI).

	\section*{Acknowledgment}

	My sincere thanks and love for my parents Dipendra Kali Mitra and Bharati Mitra
	for their continuous inspiration, encouragement, love, patience and support
	during this software development.

	This software was designed during my B.Tech days when I was studying 8085
	Microprocessor subject itself. Since then it has evolved and attained much maturity.
	I would do injustice if I do not mention the name of my friend circle, who always
	maintained a positive vibe and joyous environment for creative work culture.
	Cheers to my college friends Anirban Goswami, Debanjan Chatterjee and Abhyuday
	Jatty.

	I salute the spontaneous guidance and inspiration of my college faculty members
	Amitava Hatial, Saibal Dutta, and Surajit Bagchi.

	\section*{Contact Details}

	In the end I would love to request my esteemed users to kindly send their
	valuable suggestions for the improvement of the software and to notify me any
	errors that you may come across while using the software. You can comment in the
	blogspot \url{http://8085simulatorj.blogspot.in}. If you need to contact me
	directly just drop a mail in my mailbox, \href{mailto:jm61288@gmail.com}{jm61288@gmail.com}.
	If it is applicable for all users then I would suggest you to post it in the
	blogspot, so that it is accessible to other users as well.

	{ \vspace{0.1\linewidth} \raggedleft \makebox[0.25\linewidth][l]{Jubin Mitra}\\
	% \makebox[0.25\linewidth][l]{EMAIL: \href{mailto:jm61288@gmail.com}{jm61288@gmail.com}}
	}

	\tableofcontents

	\clearpage

	\pagenumbering{arabic}

	\chapter*{License and Disclaimer}

	\section*{GNU General Public License version 2 (GPLv2)}

	Copyright (C) 1989, 1991 Free Software Foundation, Inc. 59 Temple Place, Suite
	330, Boston, MA 02111-1307 USA

	Everyone is permitted to copy and distribute verbatim copies of this license document,
	but changing it is not allowed.

	\section*{Preamble}

	The licenses for most software are designed to take away your freedom to share
	and change it. By contrast, the GNU General Public License is intended to guarantee
	your freedom to share and change free software--to make sure the software is
	free for all its users. This General Public License applies to most of the Free
	Software Foundation's software and to any other program whose authors commit to
	using it. (Some other Free Software Foundation software is covered by the GNU Library
	General Public License instead.) You can apply it to your programs, too.

	When we speak of free software, we are referring to freedom, not price. Our General
	Public Licenses are designed to make sure that you have the freedom to distribute
	copies of free software (and charge for this service if you wish), that you
	receive source code or can get it if you want it, that you can change the software
	or use pieces of it in new free programs; and that you know you can do these
	things.

	To protect your rights, we need to make restrictions that forbid anyone to deny
	you these rights or to ask you to surrender the rights. These restrictions
	translate to certain responsibilities for you if you distribute copies of the
	software, or if you modify it.

	For example, if you distribute copies of such a program, whether gratis or for
	a fee, you must give the recipients all the rights that you have. You must make
	sure that they, too, receive or can get the source code. And you must show
	them these terms so they know their rights.

	We protect your rights with two steps: (1) copyright the software, and (2)
	offer you this license which gives you legal permission to copy, distribute and/or
	modify the software.

	Also, for each author's protection and ours, we want to make certain that everyone
	understands that there is no warranty for this free software. If the software
	is modified by someone else and passed on, we want its recipients to know that
	what they have is not the original, so that any problems introduced by others
	will not reflect on the original authors' reputations.

	Finally, any free program is threatened constantly by software patents. We wish
	to avoid the danger that redistributors of a free program will individually obtain
	patent licenses, in effect making the program proprietary. To prevent this, we
	have made it clear that any patent must be licensed for everyone's free use or
	not licensed at all.

	The precise terms and conditions for copying, distribution and modification follow.

	\section*{TERMS AND CONDITIONS FOR COPYING, DISTRIBUTION AND MODIFICATION}
	\textbf{0.} This License applies to any program or other work which contains a
	notice placed by the copyright holder saying it may be distributed under the
	terms of this General Public License. The "Program", below, refers to any such
	program or work, and a "work based on the Program" means either the Program or
	any derivative work under copyright law: that is to say, a work containing the
	Program or a portion of it, either verbatim or with modifications and/or translated
	into another language. (Hereinafter, translation is included without
	limitation in the term "modification".) Each licensee is addressed as "you".

	Activities other than copying, distribution and modification are not covered by
	this License; they are outside its scope. The act of running the Program is not
	restricted, and the output from the Program is covered only if its contents
	constitute a work based on the Program (independent of having been made by running
	the Program). Whether that is true depends on what the Program does.

	\textbf{1.} You may copy and distribute verbatim copies of the Program's
	source code as you receive it, in any medium, provided that you conspicuously
	and appropriately publish on each copy an appropriate copyright notice and
	disclaimer of warranty; keep intact all the notices that refer to this License
	and to the absence of any warranty; and give any other recipients of the
	Program a copy of this License along with the Program.

	You may charge a fee for the physical act of transferring a copy, and you may at
	your option offer warranty protection in exchange for a fee.

	\textbf{2.} You may modify your copy or copies of the Program or any portion
	of it, thus forming a work based on the Program, and copy and distribute such
	modifications or work under the terms of Section 1 above, provided that you also
	meet all of these conditions:

	\textbf{a)} You must cause the modified files to carry prominent notices
	stating that you changed the files and the date of any change. \textbf{ b) }You
	must cause any work that you distribute or publish, that in whole or in part
	contains or is derived from the Program or any part thereof, to be licensed as
	a whole at no charge to all third parties under the terms of this License.

	\textbf{c)} If the modified program normally reads commands interactively when
	run, you must cause it, when started running for such interactive use in the
	most ordinary way, to print or display an announcement including an appropriate
	copyright notice and a notice that there is no warranty (or else, saying that you
	provide a warranty) and that users may redistribute the program under these
	conditions, and telling the user how to view a copy of this License. (Exception:
	if the Program itself is interactive but does not normally print such an
	announcement, your work based on the Program is not required to print an announcement.)

	These requirements apply to the modified work as a whole. If identifiable sections
	of that work are not derived from the Program, and can be reasonably
	considered independent and separate works in themselves, then this License,
	and its terms, do not apply to those sections when you distribute them as separate
	works. But when you distribute the same sections as part of a whole which is a
	work based on the Program, the distribution of the whole must be on the terms of
	this License, whose permissions for other licensees extend to the entire whole,
	and thus to each and every part regardless of who wrote it.

	Thus, it is not the intent of this section to claim rights or contest your
	rights to work written entirely by you; rather, the intent is to exercise the
	right to control the distribution of derivative or collective works based on
	the Program.

	In addition, mere aggregation of another work not based on the Program with the
	Program (or with a work based on the Program) on a volume of a storage or distribution
	medium does not bring the other work under the scope of this License.

	\textbf{3.} You may copy and distribute the Program (or a work based on it,
	under Section 2) in object code or executable form under the terms of Sections
	1 and 2 above provided that you also do one of the following:

	\textbf{a)} Accompany it with the complete corresponding machine-readable
	source code, which must be distributed under the terms of Sections 1 and 2 above
	on a medium customarily used for software interchange; or,

	\textbf{b)} Accompany it with a written offer, valid for at least three years,
	to give any third party, for a charge no more than your cost of physically
	performing source distribution, a complete machine-readable copy of the corresponding
	source code, to be distributed under the terms of Sections 1 and 2 above on a
	medium customarily used for software interchange; or,

	\textbf{c)} Accompany it with the information you received as to the offer to
	distribute corresponding source code. (This alternative is allowed only for
	noncommercial distribution and only if you received the program in object code
	or executable form with such an offer, in accord with Subsection b above.)

	The source code for a work means the preferred form of the work for making
	modifications to it. For an executable work, complete source code means all
	the source code for all modules it contains, plus any associated interface definition
	files, plus the scripts used to control compilation and installation of the
	executable. However, as a special exception, the source code distributed need not
	include anything that is normally distributed (in either source or binary form)
	with the major components (compiler, kernel, and so on) of the operating system
	on which the executable runs, unless that component itself accompanies the
	executable.

	If distribution of executable or object code is made by offering access to
	copy from a designated place, then offering equivalent access to copy the source
	code from the same place counts as distribution of the source code, even
	though third parties are not compelled to copy the source along with the
	object code.

	\textbf{4.} You may not copy, modify, sublicense, or distribute the Program
	except as expressly provided under this License. Any attempt otherwise to copy,
	modify, sublicense or distribute the Program is void, and will automatically
	terminate your rights under this License. However, parties who have received
	copies, or rights, from you under this License will not have their licenses
	terminated so long as such parties remain in full compliance.

	\textbf{5.} You are not required to accept this License, since you have not
	signed it. However, nothing else grants you permission to modify or distribute
	the Program or its derivative works. These actions are prohibited by law if you
	do not accept this License. Therefore, by modifying or distributing the Program
	(or any work based on the Program), you indicate your acceptance of this
	License to do so, and all its terms and conditions for copying, distributing
	or modifying the Program or works based on it.

	\textbf{6.} Each time you redistribute the Program (or any work based on the
	Program), the recipient automatically receives a license from the original
	licensor to copy, distribute or modify the Program subject to these terms and conditions.
	You may not impose any further restrictions on the recipients' exercise of the
	rights granted herein. You are not responsible for enforcing compliance by
	third parties to this License.

	\textbf{7.} If, as a consequence of a court judgment or allegation of patent
	infringement or for any other reason (not limited to patent issues), conditions
	are imposed on you (whether by court order, agreement or otherwise) that
	contradict the conditions of this License, they do not excuse you from the conditions
	of this License. If you cannot distribute so as to satisfy simultaneously your
	obligations under this License and any other pertinent obligations, then as a consequence
	you may not distribute the Program at all. For example, if a patent license would
	not permit royalty-free redistribution of the Program by all those who receive
	copies directly or indirectly through you, then the only way you could satisfy
	both it and this License would be to refrain entirely from distribution of the
	Program.

	If any portion of this section is held invalid or unenforceable under any
	particular circumstance, the balance of the section is intended to apply and the
	section as a whole is intended to apply in other circumstances.

	It is not the purpose of this section to induce you to infringe any patents or
	other property right claims or to contest validity of any such claims; this
	section has the sole purpose of protecting the integrity of the free software
	distribution system, which is implemented by public license practices. Many
	people have made generous contributions to the wide range of software
	distributed through that system in reliance on consistent application of that
	system; it is up to the author/donor to decide if he or she is willing to distribute
	software through any other system and a licensee cannot impose that choice.

	This section is intended to make thoroughly clear what is believed to be a consequence
	of the rest of this License.

	\textbf{8.} If the distribution and/or use of the Program is restricted in
	certain countries either by patents or by copyrighted interfaces, the original
	copyright holder who places the Program under this License may add an explicit
	geographical distribution limitation excluding those countries, so that
	distribution is permitted only in or among countries not thus excluded. In such
	case, this License incorporates the limitation as if written in the body of
	this License.

	\textbf{9.} The Free Software Foundation may publish revised and/or new versions
	of the General Public License from time to time. Such new versions will be
	similar in spirit to the present version, but may differ in detail to address new
	problems or concerns.

	Each version is given a distinguishing version number. If the Program
	specifies a version number of this License which applies to it and "any later version",
	you have the option of following the terms and conditions either of that version
	or of any later version published by the Free Software Foundation. If the
	Program does not specify a version number of this License, you may choose any version
	ever published by the Free Software Foundation.

	\textbf{10.} If you wish to incorporate parts of the Program into other free
	programs whose distribution conditions are different, write to the author to ask
	for permission. For software which is copyrighted by the Free Software
	Foundation, write to the Free Software Foundation; we sometimes make
	exceptions for this. Our decision will be guided by the two goals of preserving
	the free status of all derivatives of our free software and of promoting the sharing
	and reuse of software generally.

	\section*{NO WARRANTY}

	11. BECAUSE THE PROGRAM IS LICENSED FREE OF CHARGE, THERE IS NO WARRANTY FOR THE
	PROGRAM, TO THE EXTENT PERMITTED BY APPLICABLE LAW. EXCEPT WHEN OTHERWISE STATED
	IN WRITING THE COPYRIGHT HOLDERS AND/OR OTHER PARTIES PROVIDE THE PROGRAM "AS
	IS" WITHOUT WARRANTY OF ANY KIND, EITHER EXPRESSED OR IMPLIED, INCLUDING, BUT
	NOT LIMITED TO, THE IMPLIED WARRANTIES OF MERCHANTABILITY AND FITNESS FOR A PARTICULAR
	PURPOSE. THE ENTIRE RISK AS TO THE QUALITY AND PERFORMANCE OF THE PROGRAM IS
	WITH YOU. SHOULD THE PROGRAM PROVE DEFECTIVE, YOU ASSUME THE COST OF ALL
	NECESSARY SERVICING, REPAIR OR CORRECTION.

	12. IN NO EVENT UNLESS REQUIRED BY APPLICABLE LAW OR AGREED TO IN WRITING WILL
	ANY COPYRIGHT HOLDER, OR ANY OTHER PARTY WHO MAY MODIFY AND/OR REDISTRIBUTE
	THE PROGRAM AS PERMITTED ABOVE, BE LIABLE TO YOU FOR DAMAGES, INCLUDING ANY
	GENERAL, SPECIAL, INCIDENTAL OR CONSEQUENTIAL DAMAGES ARISING OUT OF THE USE
	OR INABILITY TO USE THE PROGRAM (INCLUDING BUT NOT LIMITED TO LOSS OF DATA OR DATA
	BEING RENDERED INACCURATE OR LOSSES SUSTAINED BY YOU OR THIRD PARTIES OR A FAILURE
	OF THE PROGRAM TO OPERATE WITH ANY OTHER PROGRAMS), EVEN IF SUCH HOLDER OR OTHER
	PARTY HAS BEEN ADVISED OF THE POSSIBILITY OF SUCH DAMAGES.

	\section*{Disclaimer}
	This is a voluntary work of an individual to develop a common platform for 8085
	programming. Please be advised that nothing found here has necessarily been
	peer reviewed by people with the expertise required to provide you with
	complete, accurate or reliable information. So, user's discretion is advisable.

	That is not to say that you will not always find inaccurate results in 8085 Simulator;
	but sometimes due to bug you may get some. However, the author cannot
	guarantee the validity or the liability of the results found using this
	software.

	\chapter{Product Description}

	\section{Motivation}
	Understanding of Intel 8085 microprocessor is fundamental to getting insight
	into the Von-Neumann Architecture. It was first introduced in 1976, since then
	many generations of computer architecture have come up, some still persists while
	others are lost in history. This microprocessor still survives because it is
	still popular in university and training institutes to get students acquainted
	with basic computer architecture. For this purpose 8085 trainer kit are available
	on the market. However, with more popular technologies to learn, technical
	syllabus has very low time bandwidth available for this topic. All that is necessary
	for the students is to understand the functional working model of this basic architecture
	and then proceed on to next advance level of the subject.

	With this academic learning purpose in mind this simulator software is designed.
	It helps in get started easily with example codes, and to learn the architecture
	playfully. It also provides a trainer kit as an appealing functional
	alternative to real hardware. The users can write assembly code easily and get
	results quickly without even having the actual hardware.

	\section{Installation and Upgrade Note}

	The program code is written Java Syntax and available in java virtual machine
	executable format (.jar). To run in :\\ \textbf{Windows : }\\ 1) Make sure you
	have Java installed on your system. Check this by typing \textbf{java -version}
	into the command terminal. If you don't have the latest version of Java,
	update it before proceeding. \\ 2) Install Java (ver >6) \url{http://www.java.com/en/download/manual.jsp}\\
	3) Just \textbf{double clic}k the ``.jar '' file, it should execute. \\ 4)
	Otherwise you can execute in CMD ( Command Prompt ) by typing `` \textbf{java
	-jar <filename>.jar} ''\\ \\ \textbf{Linux : }\\ 1) Open terminal and type ``
	\textbf{java -jar <filename>.jar} '' \\ \\ \textbf{\large UPDATES : }\\
	Automatic or push updates are not supported in this software. Users are requested
	to keep track of the new release available at the web-link : \url{https://8085simulator.codeplex.com/}.
	\section{Limitations}

	This or any 8085 simulator software is no way a replacement for real hardware.
	It only does functional simulation of the codes. It is not an emulator and hence
	do not expect that the timing information will be accurately modeled. However,
	the exact performance of the code can only be monitored in real 8085 microprocessor
	hardware.

	\section{Known Issues}
	\begin{itemize}
		\item Issue 1 : DAA instruction wrongly toggles the carry flag if already there
			is a carry instead of setting it high, like take for example (88H + 88H). Users
			need to be cautious while using this instruction. It will be fixed in
			future realize v2.1.

		\item Issue 2: In Assembler Window, during pre-processing stage of the code it
			flags error if\\ ` ; ' (SEMICOLON) comment marking character is followed
			after `` // '' (DOUBLE FORWARD SLASH).\\ Example $\rightarrow$ "<Label>: <Assembler
			Code> // <Comments> ; <More Comments>"
	\end{itemize}

	\section{Software Design Architecture}

	\begin{figure}[htbp]
		\centering
		\includegraphics[width=0.9\linewidth]{./soft_arch.png}
		\caption{Software Architecture}
		\label{fig:software:architecture}
	\end{figure}

	\subsection{Preprocessor}
	Assembler directives are lines included in the code of programs preceded by a
	hash sign (\#). These lines are not program statements but directives for the preprocessor.
	The preprocessor examines the code before actual assembling of code begins and
	resolves all these directives before any code is actually generated by regular
	statements.

	\subsection{Assembler}
	It uses a 2 pass Assembler. In first pass it constructs \textit{the symbol
	table} in which every label of the assembly program is stored with its
	corresponding location. In the second pass the assembler locates (using the
	flags array) and completes (using the symbol table) the partial mnemonics instructions.
	It then convert Mnemonic to Opcode using a mapping method.

	\subsection{Simulator Engine}
	It resets all the register. Then starts from "Origin address". It scans the
	opcode value and sends it to "Opcode to instruction set look table". It then instructs
	the simulator engine the registers that will be affected, the number of data
	opcode that follows after the instruction opcode to increment the address and
	also to increment the number of clock cycles accordingly.

	\subsection{Step-wise Traversal Controller}
	It consists of \textit{memory snapshot maker} and \textit{memory register -
	data value monitor}. During forward traversal \textit{memory snapshot maker}
	dumps the entire memory current values to a temp file. With each forward step one
	temp file is created in the working directory pool of the software. During
	backward traversal the \textit{memory snapshot maker} read backs the temp files
	and reloads with the past value. Once the process is stopped it clears out all
	the snapshots that are dumped. In this manner this software can able to traverse
	also in backward direction, inspite of using forward traversal instruction
	code.

	\section{Source Code}
	The entire design is built in \textbf{Netbeans IDE} with JDK bundle. It can
	easily be opened by the software. The coding was done in bit unprofessional way,
	as it was developed during very early stage of my academics. Students are free
	to use the code for their understanding and distribution as defined under the GNU
	license agreement.

	It is being actively maintained in GIT repository find it at link :\\
	\url{https://8085simulator.codeplex.com/SourceControl/latest}\\ \url{https://github.com/8085simulator/8085simulator}

	\chapter{Features}
	\begin{enumerate}
		\item Assembler Editor
			\begin{itemize}
				\item Can load Programs written in other simulator

				\item Auto-correct and auto-indent features

				\item Supports assembler directives

				\item Number parameters can be given in binary, decimal and hexadecimal
					format

				\item Supports writing of comments

				\item Supports labeling of instructions, even in macros

				\item Has error checking facility

				\item Syntax Highlighting
			\end{itemize}

		\item Disassembler Editor
			\begin{itemize}
				\item Supports loading of Intel specific hex file format

				\item It can successfully reverse trace the original program from the assembly
					code, in most of the cases

				\item Syntax Highlighting and Auto Spacing
			\end{itemize}

		\item Assembler Workspace
			\begin{itemize}
				\item Contains the Address field, Label, Mnemonics, Hex-code, Mnemonic Size,
					M-Cycles and T-states

				\item Static Timing diagram of all instruction sets are supported

				\item Dynamic Timing diagram during step by step simulation

				\item It has error checking facility also
			\end{itemize}

		\item Memory Editor
			\begin{itemize}
				\item Can directly update data in a specified memory location

				\item It has 3 types of interface, user can choose from it according to
					his need.
					\begin{itemize}
						\item Show entire memory content

						\item Show only loaded memory location

						\item Store directly to specified memory location
					\end{itemize}

				\item Allows user to choose memory range
			\end{itemize}

		\item I/O Editor
			\begin{itemize}
				\item It is necessary for peripheral interfacing.

				\item Enables direct editing of content
			\end{itemize}

		\item Interrupt Editor
			\begin{itemize}
				\item All possible interrupts are supported. Interrupts are triggered by
					pressing the appropriate column (INTR, TRAP, RST 7.5, RST 6.5, RST 5.5)
					on the interrupt table. The simulation can be reset any time by pressing
					the clear memory in the settings tab.
			\end{itemize}

		\item Debugger
			\begin{itemize}
				\item Support of breakpoints

				\item Step by step execution/debugging of program.

				\item It supports both forward and backward traversal of programs.

				\item Allows continuation of program from the breakpoint.
			\end{itemize}

		\item Simulator
			\begin{itemize}
				\item There are 3 level of speed for simulation:
					\begin{itemize}
						\item Step-by-step $\longrightarrow$ Automatic line by line execution
							with each line highlighting. The time to halt at each line is be
							decided by the user.

						\item Normal $\longrightarrow$ Full execution with reflecting intermittent
							states periodically.

						\item Ultimate $\longrightarrow$ Full execution with reflecting final
							state directly.
					\end{itemize}

				\item There are 2 modes of simulator engine:
					\begin{itemize}
						\item Run all at a Time $\longrightarrow$ It takes the current settings
							from the simulation speed level and starts execution accordingly.

						\item Step by Step $\longrightarrow$ It is manual mode of control of
							FORWARD and BACKWARD traversal of instruction set. It also
							displays the in-line comment if available for currently executed
							instruction.
					\end{itemize}

				\item Allows setting of starting address for the simulator

				\item Users can choose the mnemonic where program execution should terminate
			\end{itemize}

		\item Helper
			\begin{itemize}
				\item Help on the mnemonics is integrated

				\item CODE WIZARD is a tool added to enable users with very little knowledge
					of assembly code could also 8085 assembly programs.

				\item Already loaded with plenty SAMPLE programs

				\item Dynamic loading of user code if placed in user\_code folder

				\item It also includes a user manual
			\end{itemize}

		\item Printing
			\begin{itemize}
				\item Assembler Content

				\item Workspace Content
			\end{itemize}

		\item Register Bank $\longrightarrow$ Each register content is accompanied with
			its equivalent binary value
			\begin{itemize}
				\item Accumulator, Reg B, Reg C, Reg D, Reg E, Reg H, Reg L, Memory (M)

				\item Flag Register

				\item Stack Pointer (SP)

				\item Memory Pointer (HL)

				\item Program Status Word (PSW)

				\item Program Counter (PC)

				\item Clock Cycle Counter

				\item Instruction Counter

				\item Special blocks for monitoring Flag register and the usage of SIM and
					RIM instruction
			\end{itemize}

		\item Crash Recovery
			\begin{itemize}
				\item Can recover programs lost due to sudden shutdown or crash of application
			\end{itemize}

		\item 8085 TRAINER KIT
			\begin{itemize}
				\item It simulates the kit as if the user is working in the lab. It
					basically uses the same simulation engine at the back-end
			\end{itemize}

		\item TOOLS
			\begin{itemize}
				\item Insert DELAY Subroutine TOOL
					\begin{itemize}
						\item It is a powerful wizard to generate delay subroutine with user
							defined delay using any sets of register for a particular operating
							frequency of 8085 microprocessor.
					\end{itemize}

				\item Interrupt Service Subroutine TOOL
					\begin{itemize}
						\item It is a handy way to set memory values at corresponding vector
							interrupt address
					\end{itemize}

				\item Number Conversion Tool
					\begin{itemize}
						\item It is a portable interconversion tool for Hexadecimal, decimal
							and binary numbers. So, that user do not need to open separate
							calculator for it.
					\end{itemize}
			\end{itemize}
	\end{enumerate}

	\chapter{Comparitive Analysis}

	\begin{table}[H]
		\centering
		\caption{The Comparitive analysis between different softwares}
		\label{table:compare} \resizebox{\linewidth}{!}{
		\begin{tabular}{|c|c|c|c|c|c|}
			\hline
			                                          &                         &                        &                     &                      &                     \\
			                                          & \textbf{8085 Simulator} & \textbf{Osonsoft 8085} & \textbf{GNUSim8085} & \textbf{Vaneet 8085} & \textbf{Abhijit's } \\
			\textbf{Features}                         & \textbf{version 2.0}    & \textbf{simulator}     & \textbf{simulator}  & \textbf{simulator}   & \textbf{8085}       \\
			                                          & \textbf{(Jubin's)}      &                        &                     &                      & \textbf{simulator}  \\
			                                          &                         &                        &                     &                      &                     \\
			\hline
			\textbf{Platform Independent}             & $\bigstar$              &                        &                     &                      &                     \\
			\hline
			\textbf{Backward Traversal Feature}       & $\bigstar$              &                        &                     &                      &                     \\
			\hline
			\textbf{8085 Trainer Kit Simulation}      & $\bigstar$              &                        &                     &                      & $\blacklozenge$     \\
			\hline
			\textbf{Backward Traversal Feature}       & $\bigstar$              &                        &                     &                      &                     \\
			\hline
			\textbf{Simulation speed control}         & $\bigstar$              & $\blacklozenge$        &                     &                      &                     \\
			\hline
			\textbf{Number Conversion Tool}           & $\bigstar$              &                        &                     &                      &                     \\
			\hline
			\textbf{Setting of memory range,}         &                         &                        &                     &                      &                     \\
			\textbf{stop mnemonic }                   & $\bigstar$              & $\bigstar$             &                     &                      &                     \\
			\textbf{and starting address}             &                         &                        &                     &                      &                     \\
			\hline
			\textbf{Delay subroutine Insertion Tool } & $\bigstar$              &                        &                     &                      &                     \\
			\hline
			\textbf{Crash recovery feature}           & $\bigstar$              &                        &                     &                      &                     \\
			\hline
			\textbf{Code Wizard}                      & $\bigstar$              & $\bigstar$             &                     &                      &                     \\
			\hline
		\end{tabular}
		} \raggedright $\blacklozenge$--Partial Support ; $\bigstar$ -- Full support
	\end{table}

	The \cref{table:compare} compares the features that are special to this simulator.
	Apart from the contents listed most features are common, except for the peripheral
	attachment which will be added in future release.

	\chapter{Assembler Directives}
	The assembler directives \cite{intel} are the instructions to the assembler
	concerning the program being assembled; they also are called \textit{pseudo
	instructions} or \textit{pseudo opcodes}.

	In the Assembler Editor, the Assembler Directives \textbf{must be preceded by }\textbf{`.'}
	or \textbf{`\#'}. The editor would then understand and would automatically change
	font foreground color to red color. Since execution of assembler directives do
	not assign any machine code but it directs the assembler engine and the memory
	loader to load a specific user code at user defined position. So it \textbf{loads
	code directly in the MEMORY EDITOR, it's output code is not visible in ASSEMBLER
	WORKSPACE}. \Cref{sec:asm:dir} lists the assembler directives that are currently
	supported by the assembler.

	\section{Directives}
	\label{sec:asm:dir}
	\begin{table}[htbp]
		\centering
		\begin{tabular}{lcll}
			   & \textbf{Assembler}  & \textbf{Example}    & \textbf{Description}                                                \\
			   & \textbf{Directives} &                     &                                                                     \\
			1. & ORG                 & \# ORG C000H        & The next block of instruction should be stored                      \\
			   & (Origin)            &                     & in memory locations starting at C000H                               \\
			   &                     &                     &                                                                     \\
			2. & BEGIN               & \# BEGIN 2000H      & To start simulation from address 2000H                              \\
			   & (Start)             &                     &                                                                     \\
			   &                     &                     &                                                                     \\
			3. & END                 & \# END              & End of Assembly. It places the mnemonic defined                     \\
			   & (Stop)              &                     & at "Settings $\rightarrow$ Stop Simulation at Mnemonic"             \\
			   &                     &                     &                                                                     \\
			4. & EQU                 & \# OUTBUF EQU 3945H & The value of the label OUTBUF is 3945H.                             \\
			   & (Equal)             &                     & This may be used as memory location.                                \\
			   &                     &                     &                                                                     \\
			5. & DB                  & \# DATA: DB F5H,12H & Initializes an area byte by byte,in successive memory locations     \\
			   & (Define Byte)       &                     & until all values are stored. Label DATA stores the initial address. \\
			   &                     &                     &                                                                     \\
			6. & DW                  & \# LABEL: DW 2050H  & Initializes an area two bytes at a time.                            \\
			   & (Define Word)       &                     &                                                                     \\
			   &                     &                     &                                                                     \\
			7. & DS                  & \# STACK: DS 4      & Reserves a specified number of memory locations and set             \\
			   & (Define Storage)    &                     & the initial address to label STACK.
		\end{tabular}
	\end{table}

	\pagebreak
	\section{Number Format Support}
	The Assembler for both code and assembler directive support flexible number
	entry mode

	\paragraph{Binary Number Entry Format}
	\begin{itemize}
		\item Digits should consists of 1's and 0's.

		\item The number of digits must be greater than 4, to prevent confusion with
			default Hexadecimal mode.

		\item The number must be followed by character 'b' or 'B', to indicate that
			it is a binary number.\\ \textit{Example:} To enter ``F''(Hexadecimal Number)
			write it as 01111\textbf{B} or 01111\textbf{b}
	\end{itemize}

	\paragraph{Decimal Number Entry Format}
	\begin{itemize}
		\item Digits should be within 0-9.

		\item The number of digits must be greater than 4, to prevent confusion with
			default Hexadecimal mode.

		\item The number must be followed by character 'd' or 'D', to indicate that
			it is a decimal number.\\ \textit{Example:} To enter ``F''(Hexadecimal Number)
			write it as 0015\textbf{D} or 0015\textbf{d}
	\end{itemize}

	\paragraph{Hexadecimal Number Entry Format}
	\begin{itemize}
		\item Digits should be within 0-9 and A-F.

		\item The number of digits can be of any size

		\item The number may be followed by character 'h' or 'H', to indicate that
			it is a hexadecimal number.\\ \textit{Example:} To enter ``F''(Hexadecimal
			Number) write it as 0F\textbf{H} or 0F\textbf{h} or just 0F
	\end{itemize}

	\chapter{Disassembler}

	A disassembler is a computer program that translates machine language into assembly
	language-the inverse operation to that of an assembler. A disassembler differs
	from a decompiler, which targets a high-level language rather than an assembly
	language. Disassembly, the output of a disassembler, is often formatted for
	human-readability rather than suitability for input to an assembler, making it
	principally a \textbf{reverse-engineering tool}.

	\section{Disassembler Demonstration}

	\Cref{fig:s1asm} shows a sample program i.e. "\textit{1's COMPLEMENT OF A 16-BIT
	NUMBER}" loaded in the assembly language editor. It is then assembled by
	pressing the \textbf{Assemble} button. After assembling, memory content and assembler
	workspace are shown in \cref{fig:s1asmMem} and \cref{fig:s1asmWork} respectively.
	Then the \textbf{Hexcode} is saved by selecting "FILE$\rightarrow$Save Hexcode"
	or presing "ALT+S". \\

	The generated hexcode is now loaded in the Disassembler editor by selecting "FILE$\rightarrow$Load
	Hexcode" or presing "ALT+O". As it can be seen in \cref{fig:s1hex} the Intel
	Hex formatted code is syntactically highlighted. Now, press the \textbf{Disassemble}.
	If there is some error in the code that line will be highlighted in red. The tabbed
	window will not automatically change, even if there is no error. Now open the assembler
	editor the code is disassembled, as given in \cref{fig:s1dis}. Simultaneously the
	memory content is also loaded which is same as shown in \cref{fig:s1asmMem}.
	But, it is to be remembered that assembler workspace will remain empty, until
	the code is assembled from the assembler editor.

	\begin{figure}[htbp]
		\centering
		\begin{subfigure}
			[b]{0.4\textwidth}
			\includegraphics[width=\textwidth]{sampleCode1}
			\caption{Assembled Code}
			\label{fig:s1asm}
		\end{subfigure}
		\begin{subfigure}
			[b]{0.3\textwidth}
			\includegraphics[width=\textwidth]{sampleCode1Rev}
			\caption{Disassembled Code}
			\label{fig:s1dis}
		\end{subfigure}
		\\
		\vspace{20pt}
		\begin{subfigure}
			[b]{0.5\textwidth}
			\includegraphics[width=\textwidth]{sampleCode1Asm}
			\caption{Assembler Workspace after Assembling}
			\label{fig:s1asmWork}
		\end{subfigure}
		\begin{subfigure}
			[b]{0.4\textwidth}
			\includegraphics[width=\textwidth]{sampleCode1Mem}
			\caption{Memory Content after Assembling}
			\label{fig:s1asmMem}
		\end{subfigure}
		\\
		\vspace{20pt}
		\begin{subfigure}
			[b]{0.5\textwidth}
			\includegraphics[width=\textwidth]{sampleCode1Hex}
			\caption{Hexcode of Assembled code}
			\label{fig:s1hex}
		\end{subfigure}
		\\
		\vspace{20pt}
		\caption{Showing working of Disassembler}
	\end{figure}

	\clearpage

	\section{Intel HEX}

	Intel HEX\cite{hex} is a file format for conveying binary information for
	programming 8085 microprocessor. The assembler converts the program's assembly
	language code to machine code and outputs it into a HEX file. That file is
	then imported by a programmer to "burn" the machine code to the 8085 target
	system for loading and execution.

	\begin{figure}[H]
		\centering
		\includegraphics{"./disassemble"}
		\begin{tabular}{cl}
			{\color[HTML]{999900}\rule{0.5cm}{0.5cm}} & Start Code  \\
			{\color[HTML]{009900}\rule{0.5cm}{0.5cm}} & Byte count  \\
			{\color[HTML]{4C0099}\rule{0.5cm}{0.5cm}} & Address     \\
			{\color[HTML]{990099}\rule{0.5cm}{0.5cm}} & Record type \\
			{\color[HTML]{009999}\rule{0.5cm}{0.5cm}} & Data        \\
			{\color[HTML]{999999}\rule{0.5cm}{0.5cm}} & Checksum
		\end{tabular}

		\caption[width=0.75\linewidth]{Intel HEX file format}
		\label{fig:hex}
	\end{figure}

	Each line of Intel HEX fileconsists of six parts :
	\begin{itemize}
		\item \textbf{Start code}, one character, an ASCII colon ':'.

		\item \textbf{Byte count}, two hex digits, a number of bytes (hex digit pairs)
			in the data field. 16 (0x10) or 32 (0x20) bytes of data are the usual compromise
			values between line length and address overhead.

		\item \textbf{Address}, four hex digits, a 16-bit address of the beginning
			of the memory position for the data.

		\item \textbf{Record type}, two hex digits, 00 to 05, defining the type of the
			data field.
			\begin{itemize}
				\item \textbf{00}, data record, contains data and 16-bit address.

				\item \textbf{01}, End Of File record. Must occur exactly once per file
					in the last line of the file. The byte count is 00 and the data field is
					empty. Usually the address field is also 0000, in which case the complete
					line is ':00000001FF'.
			\end{itemize}

		\item \textbf{Data}, a sequence of n bytes of the data themselves,
			represented by 2n hex digits.

		\item \textbf{Checksum}, two hex digits - the least significant byte of the two's
			complement of the sum of the values of all fields except fields 1 and 6 (Start
			code ":" byte and two hex digits of the Checksum). It is calculated by
			adding together the hex-encoded bytes (hex digit pairs), then leaving only
			the least significant byte of the result, and making a 2's complement (either
			by subtracting the byte from 0x100, or inverting it by XOR-ing with 0xFF
			and adding 0x01). If you are not working with 8-bit variables, you must suppress
			the overflow by AND-ing the result with 0xFF. The overflow may occur since
			both 0x100-0 and (0x00 XOR 0xFF)+1 equal 0x100. If the checksum is correctly
			calculated, adding all the bytes (the Byte count, both bytes in Address, the
			Record type, each Data byte and the Checksum) together will always result in
			a value wherein the least significant byte is zero (0x00). \\For example,
			on :0300300002337A1E\\ 03 + 00 + 30 + 00 + 02 + 33 + 7A = E2, 2's complement
			is 1E
	\end{itemize}

	\clearpage

	\section{Writing Hexcode in Disassembler}
	\begin{itemize}
		\item \textbf{STEP 1:} To Enter the hexcode\\

			\begin{tabular}{cccccc}
				\textbf{<Start Code>} & \textbf{<Byte Count>} & \textbf{<Address>} & <\textbf{Record Type>} & \textbf{<Data>}        & \textbf{<Checksum>} \\
				:                     & 10                    & 0000               & 00                     & Enter 10 bytes         & <ctrl+space>        \\
				                      &                       &                    &                        & of data in Hexadecimal &                     \\
				                      &                       &                    &                        & format                 &
			\end{tabular}

		\item \textbf{STEP 2:} To mark end of file\\

			\begin{tabular}{cccccc}
				\textbf{<Start Code>} & \textbf{<Byte Count>} & \textbf{<Address>} & <\textbf{Record Type>} & \textbf{<Data>} & \textbf{<Checksum>} \\
				:                     & 00                    & 0000               & 01                     &                 & FF
			\end{tabular}
	\end{itemize}

	\paragraph{TOOLS EMBEDDED IN DISASSEMBLER EDITOR}
	\begin{itemize}
		\item \textbf{\textit{AUTO CHECKSUM GENERATION}} \\ Just press \textbf{CTRL+SPACE}
			at the end of each line it is auto calculated and appended to that line

		\item \textbf{\textit{AUTO SYNTAX HIGHLIGHTING and FORMATING}} \\It is activated
			on pressing of \textbf{ENTER} key.
	\end{itemize}

	\subsection{Limitation of disassembler}
	\begin{itemize}
		\item Cannot determine the begin address of execution

		\item Fails to distinguish between user defined data code and opcode, so it
			by default decode all as opcode.
	\end{itemize}

	\chapter{Timing Diagram generator}
	The 8085 Microprocessor is designed to execute 74 different types of
	instruction. Each instruction has two parts: OPCODE (operation code) and
	OPERAND. Each functions are divided into machine cycles and each cycles is further
	divided into T-states.

	Basically, the microprocessor external communication functions can be divided
	into 3 categories of Machine Cycle:
	\begin{enumerate}
		\item Memory Read and Write

		\item I/O Read and Write

		\item Request Acknowledge
	\end{enumerate}

	Of which Request Acknowledge machine cycle is not yet supported in this
	version of the software, but internally it is simulated. \\ There are three
	methods of Timing Diagram Generation :
	\begin{enumerate}
		\item Static Timing Diagram Generation

		\item Dynamic Timing Diagram Generation
			\begin{enumerate}
				\item By Manual Step by Step Simulation

				\item By Automatic Step by Step Simulation
			\end{enumerate}
	\end{enumerate}
	\pagebreak
	\section{Static Timing Diagram Generation}
	To open the Timing Diagram window click the filled rows of the column named "T-states"
	in the Assembler workspace, as shown in \cref{fig:tstate:before:editor}.
	Static Timing Diagram is basically the machine cycles of the instruction in
	pre-simulation state. As, can be seen in \cref{fig:tstate:before:diagram}
	where default values(00H in this case) are loaded during the Memory Read Cycle
	of "LDA 1234H" from address 1234H.
	\begin{figure}[htbp]
		\centering
		\includegraphics[width=0.75\linewidth]{./TStateEditorBefore}
		\caption{The Red box marks the area of the Assembler workspace to be clicked
		before execution of the program}
		\label{fig:tstate:before:editor}
	\end{figure}
	\begin{figure}[htbp]
		\centering
		\includegraphics[width=\linewidth]{./TStateBefore}
		\caption{Timing Diagram of "LDA 1234", where the last MEMORY READ CYCLE shows
		the value at address 1234H is set with default value 00H}
		\label{fig:tstate:before:diagram}
	\end{figure}
	\pagebreak
	\section{Dynamic Timing Diagram Generation By Manual Step by Step Simulation}
	Dynamic Timing Diagram is the machine cycles of the instruction in real time
	simulation state. The stepping to the next instruction is controlled manually by
	the user, using "Step by Step" mode of execution. Here again as shown in \cref{fig:tstate:after:editor},
	need to click on the column named "T-states" of the currently highlighted row.
	Note carefully in \cref{fig:tstate:after:diagram} that the values(67H in this case)
	are loaded in the Memory Read Cycle of "LDA 1234H" during reading of memory
	content at address 1234H.

	\begin{figure}[htbp]
		\centering
		\includegraphics[width=0.75\linewidth]{./TStateEditorAfter}
		\caption{The Red box marks the area of the Assembler workspace to be clicked
		during manual step by step simulation}
		\label{fig:tstate:after:editor}
	\end{figure}
	\begin{figure}[htbp]
		\centering
		\includegraphics[width=\linewidth]{./TStateAfter}
		\caption{Timing Diagram of "LDA 1234", where the last MEMORY READ CYCLE shows
		the value at address 1234H is loaded with 67H}
		\label{fig:tstate:after:diagram}
	\end{figure}
	\pagebreak
	\section{Dynamic Timing Diagram Generation By Automatic Step by Step
	Simulation}
	Dynamic Timing Diagram is the machine cycles of the instruction in real time
	simulation state. The stepping to the next instruction is not controlled manually
	but by the simulator itself. As in this case user need to step down the "Run all
	at a Time" simulation speed to "Step by Step " mode of execution with user defined
	delay. Initially user need to open one time Static Timing Diagram Window, then
	it opens automatically and updates during each step of execution.

	\chapter{Trainer Kit Emulator}

	The keyboard enables the user to enter and store the 8085 Hex machine code in the
	R/W memory. The seven segment display is used to display memory addresses and
	their contents while entering , monitoring or modifying the programs. It also has
	two LEDS which blink alternatively on program execution. The Graphical Trainer
	Kit can be launched from the sub-menu item ``\textbf{8085 microprocessor
	trainer kit}'' placed under menu item ``View''. Shortcut key for launching
	this Trainer Kit Emulator is \textbf{"F9"}. In the back-end it is basically
	using the same engine, so at any step user can switch between any two environments.

	\begin{figure}[htbp]
		\centering
		\includegraphics[width=0.75\linewidth]{"./kit"}
		\caption{8085 Trainer Kit}
	\end{figure}

	\section{Keyboard}
	The keyboard has 24 keys; 16 keys for the Hex digits 0 to F and remaining keys
	are used to perform various functions. Some of the Hex digit keys has dual
	function: data entry mode and register monitor mode. The function of these
	keys are described as follows:

	\begin{enumerate}
		\item 0 to F: Enter Hex digits

		\item Reset: To terminate the current execution. It does not clear register or
			memory contents. It is doing the same function as pressing STOP button
			during the execution of code in 8085 simulator workspace. To clear memory content,
			goto Settings $\rightarrow$ Clear Memory, or press "CTRL+SHIFT+DELETE".

		\item Halt: It pauses the program at any stage of execution. It is equivalent
			to pressing PAUSE during the execution of code in 8085 simulator workspace.
			From where it is possible to do both forward and backward traversal.

		\item DCR: Decrements the memory address and displays the new address and
			its data.

		\item INR: Increments the memory address and displays the new address and
			its data.

		\item SET/MEM: To enter contents in particular memory address.

		\item REG: To monitor the current register content.

		\item GO: To set the starting address of execution.

		\item EXEC: To begin execution of the program from the begin address that is
			set. It takes the simulation speed that is default by the user in the editor.
			But, the default speed is set to ultimate.
	\end{enumerate}

	\section{Using the Trainer Kit Emulator}

	\subsection{How to enter a program}
	Let's take one of the sample program as shown in \cref{fig:samplecode2}, to illustrate
	the programing process. The detailed step by step loading instruction are
	given in \cref{table:trainer:load}.
	\begin{figure}[htbp]
		\centering
		\includegraphics[width=0.75\linewidth]{sampleCode2Asm}
		\caption{Sample Program of 1's COMPLEMENT OF AN 8-BIT NUMBER}
		\label{fig:samplecode2}
	\end{figure}

	When we load a program we enter the Hex codes in memory locations for given
	instructions.\\

	{ \begin{table}[htbp]\caption{Showing the buttons to be pressed sequentially to load the program in the memory} \label{table:trainer:load} \centering \newcommand{\Ts}{\rule{0pt}{2.6ex}} % top strut
	\newcommand{\Bs}{\rule[-1.1ex]{0pt}{0pt}} % bottom strut
	\begin{tabular}{lrc}\toprule \multicolumn{3}{c}{To load the Program}\\ \midrule STEP & 1: & \boxed{RESET}\\ STEP & 2: & \Ts\boxed{SET/MEM}\Bs\\ STEP & 3: & \Ts\boxed{C} \boxed{0} \boxed{0} \boxed{0}\Bs\\ STEP & 4: & \Ts\boxed{INR}\Bs\\ STEP & 5: & \Ts\boxed{3} \boxed{A}\Bs\\ STEP & 6: & \Ts\boxed{INR}\Bs\\ STEP & 7: & \Ts\boxed{5} \boxed{0}\Bs\\ STEP & 8: & \Ts\boxed{INR}\Bs\\ STEP & 9: & \Ts\boxed{C} \boxed{0}\Bs\\ STEP & 10: & \Ts\boxed{INR}\Bs\\ STEP & 11: & \Ts\boxed{2} \boxed{F}\Bs\\ STEP & 12: & \Ts\boxed{INR}\Bs\\ STEP & 13: & \Ts\boxed{3} \boxed{2}\Bs\\ STEP & 14: & \Ts\boxed{INR}\Bs\\ STEP & 15: & \Ts\boxed{5} \boxed{1}\Bs\\ STEP & 16: & \Ts\boxed{INR}\Bs\\ STEP & 17: & \Ts\boxed{C} \boxed{0}\Bs\\ STEP & 18: & \Ts\boxed{INR}\Bs\\ STEP & 19: & \Ts\boxed{7} \boxed{6}\Bs\\ \midrule \multicolumn{3}{c}{To load a value in C050}\\ \midrule STEP & 20: & \Ts\boxed{SET/MEM}\Bs\\ STEP & 21: & \Ts\boxed{C} \boxed{0} \boxed{5} \boxed{0}\Bs\\ STEP & 22: & \Ts\boxed{INR}\Bs\\ STEP & 23: & \Ts\boxed{9} \boxed{6}\Bs\\ \bottomrule\end{tabular}\end{table} }

	\subsection{To Execute the Program}
	For proper execution of the program, it is needed to direct the processor to the
	starting address of the code and then begin execution, as shown in
	\cref{table:trainer:exec}.

	{ \begin{table}[htbp]\caption{Showing the buttons to be pressed for proper execution of the code} \label{table:trainer:exec} \centering \newcommand{\Ts}{\rule{0pt}{2.6ex}} % top strut
	\newcommand{\Bs}{\rule[-1.1ex]{0pt}{0pt}} % bottom strut
	\begin{tabular}{lrc}\toprule \multicolumn{3}{c}{To begin execution}\\ \midrule STEP & 1: & \boxed{RESET}\\ STEP & 2: & \Ts\boxed{GO}\Bs\\ STEP & 3: & \Ts\boxed{C} \boxed{0} \boxed{0} \boxed{0}\Bs\\ STEP & 4: & \Ts\boxed{EXEC}\Bs\\ \bottomrule\end{tabular}\end{table} }

	\subsection{How to examine memory and register contents}
	After program execution it is essential to examine the contents of registers and
	memory. \Cref{table:trainer:monitor} lists the methods to access each and
	every register. It also lists how to examine the content at particular memory address.
	{ \begin{table}[htbp]\caption{Showing the buttons to be pressed for register and memory monitoring} \label{table:trainer:monitor} \centering \newcommand{\Ts}{\rule{0pt}{2.6ex}} % top strut
	\newcommand{\Bs}{\rule[-1.1ex]{0pt}{0pt}} % bottom strut
	\begin{tabular}{lrc}\toprule \multicolumn{3}{c}{To examine Accumulator}\\ \midrule STEP & 1: & \boxed{REG}\\ STEP & 2: & \Ts\boxed{0 A}\Bs\\ \bottomrule \toprule \multicolumn{3}{c}{To examine B}\\ \midrule STEP & 1: & \boxed{REG}\\ STEP & 2: & \Ts\boxed{1 B}\Bs\\ \bottomrule \toprule \multicolumn{3}{c}{To examine C}\\ \midrule STEP & 1: & \boxed{REG}\\ STEP & 2: & \Ts\boxed{2 C}\Bs\\ \bottomrule \toprule \multicolumn{3}{c}{To examine D}\\ \midrule STEP & 1: & \boxed{REG}\\ STEP & 2: & \Ts\boxed{3 D}\Bs\\ \bottomrule \toprule \multicolumn{3}{c}{To examine E}\\ \midrule STEP & 1: & \boxed{REG}\\ STEP & 2: & \Ts\boxed{4 E}\Bs\\ \bottomrule \toprule \multicolumn{3}{c}{To examine H}\\ \midrule STEP & 1: & \boxed{REG}\\ STEP & 2: & \Ts\boxed{5 H}\Bs\\ \bottomrule \toprule \multicolumn{3}{c}{To examine L}\\ \midrule STEP & 1: & \boxed{REG}\\ STEP & 2: & \Ts\boxed{6 L}\Bs\\ \bottomrule \toprule \multicolumn{3}{c}{To examine Memory pointed by HL}\\ \midrule STEP & 1: & \boxed{REG}\\ STEP & 2: & \Ts\boxed{7 M}\Bs\\ \bottomrule \toprule \multicolumn{3}{c}{To examine Stack Pointer}\\ \midrule STEP & 1: & \boxed{SP}\\ STEP & 2: & \Ts\boxed{8 SP}\Bs\\ \bottomrule \toprule \multicolumn{3}{c}{To examine Program Counter}\\ \midrule STEP & 1: & \boxed{PC}\\ STEP & 2: & \Ts\boxed{9 PC}\Bs\\ \bottomrule \bottomrule \toprule \multicolumn{3}{c}{To examine a Memory Address}\\ \midrule STEP & 1: & \Ts\boxed{SET/MEM}\Bs\\ STEP & 2: & \Ts\boxed{C} \boxed{0} \boxed{5} \boxed{1}\Bs\\ STEP & 3: & \Ts\boxed{INR}\Bs\\ \bottomrule\end{tabular}\end{table} }

	\clearpage

	\section{Shortcut Keys for Trainer Kit Button}
	To prevent too much switching between keyboard and mouse, some handy shortcut key
	listeners are integrated with some commonly used buttons, as listed in
	\cref{table:trainer:shortcut}. { \begin{table}[htbp]\caption{List of shortcut keys} \label{table:trainer:shortcut} \centering \newcommand{\Ts}{\rule{0pt}{2.6ex}} % top strut
	\newcommand{\Bs}{\rule[-1.1ex]{0pt}{0pt}} % bottom strut
	\begin{tabular}{lc}\toprule \textbf{Keys} & \textbf{Buttons}\\ \midrule \textbf{Esc} & RESET\\ \textbf{Alphanumeric Keys: 0-9,A-F} & HEXADECIMAL DIGITS\\ \textbf{Up-Arrow} & INR\\ \textbf{Down-Arrow} & DCR\\ \bottomrule\end{tabular}\end{table} }

	\chapter{Debugging Mode}
	The debug mode allows the user to view and/or manipulate the program's internal
	state for the purpose of debugging. The software allows step wise or block
	wise line monitor with both forward and backward traversal facility. \Cref{fig:debug:click,fig:debug:hash,fig:debug:run_time,fig:debug:pc}
	illustrates how users can use this feature in their own instruction set.
	\begin{figure}[htbp]
		\centering
		\includegraphics[width=0.9\linewidth]{./Debug_1.png}
		\caption{To Enter into debug mode click on the tick mark}
		\label{fig:debug:click}
	\end{figure}

	\begin{figure}[htbp]
		\centering
		\includegraphics[width=0.9\linewidth]{./Debug_2.png}
		\caption{Figure shows the line is now marked with ` \#' (HASH) character and
		it is ready to go into debuuging mode during execution of the line}
		\label{fig:debug:hash}
	\end{figure}

	\begin{figure}[htbp]
		\centering
		\includegraphics[width=0.9\linewidth]{./Debug_3.png}
		\caption{During run-time the simulation engine stops at marked line and
		switches to debug mode}
		\label{fig:debug:run_time}
	\end{figure}

	\begin{figure}[htbp]
		\centering
		\includegraphics[width=0.9\linewidth]{./Debug_4.png}
		\caption{On pressing ` Continue ' the simulator debugger engine steps over
		it on next halt. In the figure the changes are marked by ` Program Counter (PC)
		' register}
		\label{fig:debug:pc}
	\end{figure}

	\chapter{Tools}
	\emph{Give ordinary people the right tools, and they will design and build the
	most extraordinary things.\\ -- Neil Gershenfeld}
	\newpage
	\section{Insert Delay Subroutine}
	It is a powerful wizard to generate delay subroutine with user defined delay
	using any sets of register for a particular operating frequency of 8085
	microprocessor.

	\begin{figure}[htbp]
		\centering
		\begin{tabular}{c}
			\includegraphics[width=0.8\linewidth]{delay_sub_mark}                                        \\
			(a) Enter the marked boxes, in the order \textbf{Label}, \textbf{T-state} and \textbf{Delay} \\
			\\
			\includegraphics[width=0.5\linewidth]{delay_sub_reg}                                         \\
			(b) Showing the modes supported                                                              \\
			\\
			\includegraphics[width=1\linewidth]{delay_sub_reg_choose}                                    \\
			(c) How to choose the registers
		\end{tabular}
		\caption{The Delay Subroutine dialog box details}
	\end{figure}
	\newpage
	\subsection{Working Example of a delay sub-routine}
	\textit{Problem Statement :} \textbf{\textit{Use 3 registers to generate 10 ms
	delay in a 8085 having operating frequency of 3.072 MHz.}} \\\\ \textit{Solution::}
	The problem is solved using the tool as shown in fig. \ref{fig:delay_sub}.\\ The
	tool generated values for register $\mathbf{B = D2~H}$, $\mathbf{ C = 04 ~H}$
	and $\mathbf{ D = 01~H }$.\\ After "Run all At a Time" the
	$\mathbf{Clock~ Cycle~ Counter = 30697}$\\ The Clock Cycles taken by the delay
	subroutine is calculated by subtracting the clock cycles taken from the clock cycles
	of user written code = $\mathbf{ 30697 - 18 - 5 =30674}$.\\ Thus, the time
	taken by the delay code $= 30674 \times 326 ~ns = \mathbf{9999724~ns}$\\ Error
	Offset $= 10,000,000 - 9,999,724 = \mathbf{276 ~ns}$
	\begin{figure}[htbp]
		\centering
		\begin{tabular}{cc}
			\includegraphics[width=0.5\linewidth]{delay_sub_filled_form} & \includegraphics[width=0.5\linewidth]{delay_sub_assembler_editor} \\
			(a) Delay subroutine tool is set to the problem value        & (b) The generated code + few user added lines in the top          \\
			\\
			\includegraphics[width=0.5\linewidth]{delay_sub_compiled}    & \includegraphics[width=0.5\linewidth]{delay_sub_calc}             \\
			(c) Showing the delay of the user added code                 & (d) Total clock cycles taken by the program after Full Run        \\
			\\
		\end{tabular}
		\caption{Solution of the problem graphically}
		\label{fig:delay_sub}
	\end{figure}

	\newpage
	\section{Interrupt Service Subroutine}
	It is a handy way to set memory values at corresponding vector interrupt
	address. To invoke the tool select the option `\textbf{Subroutine}'
	$~ \rightarrow ~$ `\textbf{Interrupt Service Subroutine}'. Fig. \ref{fig:isr_routine}
	shows the steps to insert Interrupt Service Subroutine to cater to a particular
	interrupt that refers to a particular call location. In general branch
	instruction are used in interrupt call location to point to a particular
	address.
	\begin{figure}[htbp]
		\centering
		{\def\arraystretch{2} \begin{tabular}{cc}\includegraphics[width=0.5\linewidth]{isr_nop} & \includegraphics[width=0.5\linewidth]{isr_label}\\ (a) Showing the entry of TRAP Interrupt & (b) '\textit{Label1}' defined in the \textbf{assembled} code and used in entering\\ \includegraphics[width=0.5\linewidth]{isr_label_autofill} & \includegraphics[width=0.5\linewidth]{isr_memory}\\ (c) On pressing '\textit{Enter}', label is replaced by address & (d) After closing of the window, memory is updated\end{tabular} }
		\caption{Procedure to use Interrupt Service Subroutine TOOL}
		\label{fig:isr_routine}
	\end{figure}

	\newpage
	\section{Number Conversion Tool}
	It is a portable number base conversion tool between Hexadecimal, decimal and binary
	standards. It allows the developers to use the same software to calculate
	simple conversion instead of opening a new calculator and also shows all the
	value in Hexadecimal, decimal and binary format simultaneously

	\begin{figure}[htbp]
		\centering
		{\def\arraystretch{2} \begin{tabular}{cc}\includegraphics[width=0.5\linewidth]{no_tool} & \includegraphics[width=0.5\linewidth]{no_tool_select}\\ (a) When Unselected & (b) When Hexadecimal text-box is selected\\\\ \includegraphics[width=0.5\linewidth]{no_tool_value} & \includegraphics[width=0.5\linewidth]{no_tool_value_dec}\\ (c) After pressing enter on Hexadecimal text-box & (d) Decimal value of 256 entered\\\\ \includegraphics[width=0.5\linewidth]{no_tool_bin} & \\ (e) A 10-bit binary value entered for conversion &\end{tabular} }
		\caption{Using number conversion tool}
	\end{figure}

	N.B.: The tool text-boxes correctly support upto "FFFF" Hexadecimal and 8-bit binary
	full scale value.

	\printbibliography
\end{document}
